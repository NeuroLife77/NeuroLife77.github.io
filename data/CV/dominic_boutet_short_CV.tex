%%%%%%%%%%%%%%%%%%%%%%%%%%%%%%%%%%%%%%%%%
% Medium Length Graduate Curriculum Vitae
% LaTeX Template
% Version 1.1 (9/12/12)
%
% This template has been downloaded from:
% http://www.LaTeXTemplates.com
%
% Original author:
% Rensselaer Polytechnic Institute (http://www.rpi.edu/dept/arc/training/latex/resumes/)
%
% Important note:
% This template requires the res.cls file to be in the same directory as the
% .tex file. The res.cls file provides the resume style used for structuring the
% document.
%
%%%%%%%%%%%%%%%%%%%%%%%%%%%%%%%%%%%%%%%%%

%----------------------------------------------------------------------------------------
%	PACKAGES AND OTHER DOCUMENT CONFIGURATIONS
%----------------------------------------------------------------------------------------

\documentclass[margin, 10pt]{res} % Use the res.cls style, the font size can be changed to 11pt or 12pt here

\usepackage{helvet} % Default font is the helvetica postscript font
\usepackage{pifont}
\usepackage{amsmath}
\usepackage{hyperref}
%\usepackage{newcent} % To change the default font to the new century schoolbook postscript font uncomment this line and comment the one above

\setlength{\textwidth}{5.1in} % Text width of the document

\begin{document}
	
	%----------------------------------------------------------------------------------------
	%	NAME AND ADDRESS SECTION
	%----------------------------------------------------------------------------------------
	
	\moveleft.5\hoffset\centerline{\large\bf \href{https://neurolife77.github.io/}{Dominic Boutet}} % Your name at the top
	
	\moveleft\hoffset\vbox{\hrule width\resumewidth height 1pt}\smallskip % Horizontal line after name; adjust line thickness by changing the '1pt'
	
	\moveleft.5\hoffset\centerline{Montreal, Qc, Canada}
	\moveleft.5\hoffset\centerline{(438) 390-3895}
	\moveleft.5\hoffset\centerline{dominic.boutet@mail.mcgill.ca}
	
	%----------------------------------------------------------------------------------------
	
	\begin{resume}
		
		%----------------------------------------------------------------------------------------
		%	OBJECTIVE SECTION
		%----------------------------------------------------------------------------------------
		
		\section{INFO}  
		
		Ph.D. student at the IPN, McGill University. My research focuses the the modelling of multimodal neuroimaging data in the study of neurophysiological processes underlying brain activity in health and diseases (e.g., schizophrenia).
		
		%----------------------------------------------------------------------------------------
		%	EDUCATION SECTION
		%----------------------------------------------------------------------------------------
		
		\section{EDUCATION}
		
		%{\sl Diploma of College studies} \\
		%Cégep Régional de Lanaudière à l'Assomption, Qc, CA, completed May 2020 \\
		%Concentration: Natural Sciences \\
		%\vspace{-0.20cm}
		%\\%*[0.25cm]
		{\sl Bachelor of Science,} Interdisciplinary Science \\
		McGill University, Qc, CA, Completed May 2023 \\
		Concentration: Neuroscience (Major) \& Computer Science (Minor) \\
		Final cGPA: 3.98
		
		{\sl Doctor of Philosophy,} Neuroscience Science \\
		McGill University, Qc, CA, Started August 2023 \\
		Concentration: Computational Neuroscience \\
		Current cGPA: TBD
		
		\section{RESEARCH EXPERIENCE}
		
		{\sl Undergraduate Research internship} \hfill May 2021-August 2023\\The Neuro at \href{https://www.neurospeed-bailletlab.org/}{\textit{NeuroSPEED-BailletLab}}, Qc, CA\\*[0.25cm] 
		%Summer research project (2021):
		%\vspace*{0.15cm}
		%\begin{itemize} \itemsep -2pt % Reduce space between items
	%		\item[\ding{227}] Learning to use whole-brain dynamical models in combination with magnetoencephalography (MEG) data.
	%		\item Literature review, implementation of a parallel processing workflow for simulations.
%			\item Exploration of various model calibration approaches for dynamical models.
%		\end{itemize}
%		\vspace{-0.10cm}
		COMP 396 - Undergraduate Research Project class (Fall 2021):
		\vspace*{0.15cm}
		\begin{itemize} \itemsep -2pt % Reduce space between items
			\item[\ding{227}] Investigating the potential of a novel parameter space reduction metaheuristic that guides search-based optimization algorithms in high-dimensional space.
			\item Implementation of the general idea behind the metaheuristic.
			\item Implementation of a simple testing framework based on the calibration of a TVB model to MEG data.
			\item Writing of a report and preparation of a presentation for course evaluation.
		\end{itemize}
		\vspace{-0.10cm}
		\href{https://www.nserc-crsng.gc.ca/students-etudiants/ug-pc/usra-brpc_eng.asp}{\textit{NSERC USRA}} summer project (2022):
		\vspace*{0.15cm}
		\begin{itemize} \itemsep -2pt % Reduce space between items
			\item[\ding{227}] Developing a formal mathematical expression of the parameter space reduction metaheuristic designed in the previous project, implementing a flexible toolkit for its use, and thoroughly testing its efficacy against other algorithms.
			\item Writing of the API for initialization and training of neural networks used in the metaheuristic along with an approach to sampling from the parameter subspace.
			\item Implementation of accelerated simulator models of neurons and neural masses along with various search algorithms for testing. 
			\item Design and implementation of thorough testing on performance and convergence of the metaheuristic against baseline algorithms.
			\item Writing of a \href{https://arxiv.org/abs/2309.16465}{\textit{manuscript}} reporting the metaheuristic and its performance.
		\end{itemize}
		%\vspace{-0.10cm}
		%\newpage
		Undergraduate thesis, NSCI 420 (Fall 2022 - Winter 2023):
		\vspace*{0.15cm}
		\begin{itemize} \itemsep -2pt % Reduce space between items
			\item[\ding{227}] Investigating the effect of varying the number of free parameters in whole-brain dynamical models when used in a \href{https://www.nature.com/articles/s41467-021-25895-8}{\textit{neural fingerprinting}} identification task where individuals in a cohort are identified based on their brain activity.
			\item Modification of the simulation workflow of TVB models from previous projects to facilitate model calibration at varying number of free parameters.
			\item Implementation of a model calibration framework based on the parameter space reduction metaheuristic designed in a previous project.
			\item Design and implementation of hypothesis-driven tests on specific combinations of free parameter and the resulting identification accuracy. 
			\item Analysis of the results and writing of an undergraduate thesis.
		\end{itemize}
		\vspace{-0.10cm}
		\href{https://www.nserc-crsng.gc.ca/students-etudiants/ug-pc/usra-brpc_eng.asp}{\textit{NSERC USRA}} summer project (2023):
		\vspace*{0.15cm}
		\begin{itemize} \itemsep -2pt % Reduce space between items
			\item[\ding{227}] Investigating the effect of structural connectivity information in whole-brain dynamical models for \href{https://www.nature.com/articles/s41467-021-25895-8}{\textit{neural fingerprinting}}, extension of undergraduate thesis.
			\item Modification of the workflow from undergraduate thesis project to operate on individual MRI-derived structural connectivity graphs.
			\item Design and implementation of hypothesis-driven tests on the effect of structural constraints on model predictions and fingerprinting results. [Ongoing]
		\end{itemize}
	
		{\sl Research Collaboration} \hfill December 2021-June 2022\\McGill at \href{https://muhc.ca/research}{MUHC}, Qc, CA\\*[0.25cm] 
		\href{https://link.springer.com/article/10.1007/s12265-023-10474-7}{Contribution to a cardiovascular imaging data analysis project}:
		\vspace*{0.15cm}
		\begin{itemize} \itemsep -2pt % Reduce space between items
			\item[\ding{227}] Development of a feature selection and biomarker validation pipeline from oxygenation-sensitive CMR images (cardiovascular imaging).
			\item Design of a feature selection approach for biomarker identification.
			\item Design of a machine learning classifier based validation pipeline leveraging non-parameteric null testing.
			\item Writing of a manuscript, now published \href{https://link.springer.com/article/10.1007/s12265-023-10474-7}{\textit{here}}.
		\end{itemize}
	
		{\sl Doctoral projects} \hfill August 2023-Now\\The Neuro at \href{https://www.neurospeed-bailletlab.org/}{\textit{NeuroSPEED-BailletLab}}, Qc, CA\\*[0.25cm] 
		\vspace*{0.15cm}
		\textit{\textbf{Coming soon!}}
		
			
		\section{AWARDS \& \\ DISTINCTIONS}
		Academic Awards:
		\vspace*{0.15cm}
		\begin{itemize}\itemsep -2pt % Reduce space between items\\
			\item Dean's Honour List \& Faculty Of Science Scholarships Award (2021)
			\item NSERC Undergraduate Summer Research Award (2022 \& 2023)
			\item IPN Recruitment Award (2023)
			\item NSERC Canada Graduate Scholarship (2023)
			\item FRQS Doctoral Training Scholarship (2024-Now)
		\end{itemize}
	
		\section{SCIENTIFIC CONTRIBUTIONS}
		Publications:
		\begin{itemize}\itemsep -2pt % Reduce space between items\\
			\item \href{https://link.springer.com/article/10.1007/s12265-023-10474-7}{Automated Data Transformation and Feature Extraction for Oxygenation-Sensitive Cardiovascular Magnetic Resonance Images (2024) }
			\item \href{https://arxiv.org/abs/2309.16465}{A Metaheuristic for Amortized Search in High-Dimensional Parameter Spaces (2023) - [Preprint]}
		\end{itemize}
		Open Source Projects
		\vspace*{0.15cm}
		\begin{itemize}\itemsep -2pt % Reduce space between items\\
			\item \href{https://github.com/NeuroLife77/pystorm}{\textit{pystorm}}: An open source package that acts as a python port of some functions from the \href{https://neuroimage.usc.edu/brainstorm/}{\textit{brainstorm}} toolbox. It aims to be lightweight, fast (through GPU support and JIT compilation when possible), and hopefully easily scalable to high performance computing. Available through pypi as \textbf{pystorm3}.
		\end{itemize}
	
		%----------------------------------------------------------------------------------------
		%	COMMUNITY SERVICE SECTION
		%---------------------------------------------------------------------------------------- 
		
		\section{COMMUNITY \\ ENGAGEMENT}
		{\sl Undergraduate Research Lead} \hfill January 2022-May 2022\\ \href{https://yourekacanada.org/}{Youreka Canada}, CA\\*[0.25cm] 
		Acting as Principal Investigator to:
		\vspace*{0.15cm}
		\begin{itemize}\itemsep -2pt % Reduce space between items
			\item Design a complete research project based on the topic provided by Youreka.
			\item Mentor and lead a team of high school students through the whole research process, such as defining a research question, implementing a methodology, interpreting results, etc.Guiding them through the writing of a manuscript and preparation of a presentation for the Youreka Symposium.
			\item[\ding{71}] Details: We established a proof of concept for COVID-19 cases forecasting from vaccination data using time series linear models and won the Youreka Montreal Regional Finalists Award.
		\end{itemize}
		
		{\sl Vice-President of the Machine Learning Committee} \hfill August 2022-Now\\ \href{https://pharmahacks.com/}{PharmaHacks}, Qc, CA\\*[0.25cm] 
		Acting as leader within the organization to:
		\vspace*{0.15cm}
		\begin{itemize}\itemsep -2pt % Reduce space between items
			\item Help with the general operations of the organization.
			\item Help define the role for the new Machine learning committee in the organization.
			\item Evaluate the Hackathon challenges provided by our sponsors.
			\item Work with our sponsors in the development of new challenges.
			\item Design custom "PharmaHacks challenges".  
		\end{itemize}
		
		%\section{RELEVANT SKILLS}
		%Computer skills:
		%\vspace*{0.15cm}
		%\begin{itemize}\itemsep -2pt % Reduce space between items\\
		%	\item Extensive experience with relevant libraries such as numpy, pytorch, etc.
		%	\item Significant experience interacting with open source projects. %such as TVB-library and the simulation-based inference (SBI) toolkit.
		%	\item Significant knowledge of machine learning models and optimization algorithms.
		%\end{itemize}
		%\vspace{-0.10cm}
		%General skills:
		%\vspace*{0.15cm}
		%\begin{itemize}\itemsep -2pt % Reduce space between items
		%	\item Extensive knowledge of the current literature on computational models of the brain at multiple scales and model calibration algorithms.
		%	\item Great communication and teaching abilities in official and non-official settings.
		%	\item Good leadership abilities in a research or problem solving setting.
		%\end{itemize} 
		%----------------------------------------------------------------------------------------
		%	EXTRA-CURRICULAR ACTIVITIES SECTION
		%----------------------------------------------------------------------------------------
		
		%\section{EXTRA-CURRICULAR \\ ACTIVITIES} 
		
		%Elected {\it House Manager}, Rho Phi Sorority \\
		%Elected {\it Sports Chairman} \\
		%Attended Krannet Leadership Conference \\
		%Headed delegation to Rho Phi Congress \\
		%Junior varsity basketball team \\
		%Participant, seven intramural athletic teams 
		
		%----------------------------------------------------------------------------------------
		
	\end{resume}
\end{document}